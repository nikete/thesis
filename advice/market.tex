
\section{Introduction}

%cahnnel from roth economist as enginerr
%Since then, economists have gained significant experience in practical market
%design. One thing we learned from this experience is that transactions and institutions %matter at a level of detail that economists have not often had to deal with,
%and, in this respect, all markets are different. But there are also general lessons.
%This essay will consider some ways in which markets succeed and fail by looking at
%some common patterns we see of market failures, and how they have been fixed."
%3. Make it safe to participate in the market as simply as possible, as opposed to
%transacting outside of the marketplace, or engaging in strategic behavior that	 reduces
%overall welfare.


This chapter considers a subject facing a decision, who wishes to incentivize multiple experts in providing advice so as to make a choice that maximizes the reward the subject receives, while maintaining the freedom to take any action.
Experts do not have intrinsic interest in the action the subject takes, nor do they face any costs in acquiring information to advice the subject.
This model is closely related to that studied in \emph{decision markets}. The key difference being requiring that the subject of the decision maintain autonomy over what action is taken. 
It introduces \emph{advice auctions}, in which the right to provide advice and receive a share of the reward is allocated to one of the experts. This framing allows casting this setting as one of a single unit efficient allocation with interdependent valuations \citep{milgrom1982theory,maskin1992auctions,ausubel1999generalized,mclean2004informational,roughgarden2016optimal,eden2018interdependent}.
The term ``advice'' is chosen to highlight that the decision is not ultimately determined by the market, thus preserving the subject's freedom.

Our proposed mechanisms all have the same structure; in an initial stage a rights bundle over the advice and the resulting payoff is allocated efficiently. The winner form the initial stage is shown other agents reports if necessary, and selects their advice.

By leveraging the connection the mechanism design and auction literature sufficient conditions on bidders valuations can be provided such that the auction is efficient. 
An advantage of this framing as opposed to a mechanism that directly outputs a chosen action is that it allows for the expert making the recommendation to have different influences on the subject (i.e. some experts may be more persuasive).


%The subject remains in control of the decision at all points, so trying using the mechanism cannot restrict their choices or reduce their expected welfare.%\footnote{One can set a reserve price of the advice auction to the expected value of the action the agent would have picked if she did not participate in the mechanism. As long as the reserve price is set a priori or alternative reports from the non-winning bidders are the only thing used to set it, the incentives do not change from those analyzed here.}.



% Each bidder reports her type to the auctioneer. Given the reports, the auctioneer determines the allocation that maximizes surplus.
%The payment rule is the following extension of Vickrey auction pricing: a bidder is charged for a given unit that she wins according to valuations evaluated at the minimum signal that she could have reported and still won that unit.



%It is this simultaneous nature that allows us to side-step the negative (from the perspective of freedom) results that constrain sequential mechanisms to having full support over actions in order to provide aligned incentives.

%If the mechanism output was a choice to advise the subject directly, we would have to restrict the expected reward the subject receives not to depend on which expert submitted which report. 

%By having the mechanism select the expert instead of the choice, it also becomes natural to follow the Wilson doctrine in crafting a practical setting that does not require the mechanism to have access to the valuation functions. To do this

% bergemann2009information has a creference that only the indepdent signals can be detail free from 

%This closely follows work on efficiency allocation with interdependent values , and our mechanism has the same two steps: first agents submit their signals so as to be able to determine the probabilities over states of the world, and the results of that are used to pick an efficient allocation in what is private .  


\subsection{Limits to Subject Freedom In Sequential Proper Scoring Rule Based Decision Markets}

One way to incentivize the experts is by applying the machinery of prediction markets based on sequentially shared proper scoring rules to the expected reward conditional on the action.
A challenge that presents itself is how to settle the markets for the reward conditional on the action which is not taken.
One natural approach is to void the trades in the markets for these actions, this being the originally proposed mechanism in this line of work by \citet{hanson2002decision}, and only settling the markets where actions are taken.
While seemingly natural, this is not incentive compatible for the experts, even in the weak myopic sense, as shown in \citet{othman2010decision}. 

To understand why this is the case, consider a last trader facing the prediction market (sequential proper scoring rule)  where the  price is correct (matches the expected reward) for the optimal action, but there is some other action that is mispriced. The profit maximizing move for this trader is to lower the price of the optimal action below the true price of the previously mispriced action, and correct the mispriced action to its true  price. 
The utility maximizing subject would then carry out the suboptimal action, the expert would be rewarded for correctly predicting it and would receive no punishment for the error she introduced into the reward of the optimal action. 
A mechanism is called Bayes Nash Incentive Compatible (BNIC) if there is an equilibrium were every agent reports their signal truthfully and this maximizes their reward in expectation (over the state of the world).
%TODO why is this commented? bob requested an explanation!
The mechanism proposed in \citet{hanson2002decision} is not BNIC for the experts who provide advice, as witnessed by the example above, and shown in \citet{othman2010decision,chen2014eliciting}.
More generally, any sequential proper scoring rule based mechanism that is incentive compatible for the experts is incompatible with maintaining the subject's freedom to select the action that appears optimal ex-post \citep{ chen2014eliciting}. 

%Interdependent Values without Single-Crossing
%Alon Eden, Michal Feldman, Amos Fiat, Kira Goldner
%We consider a setting where an auctioneer sells a single item to n potential agents with {\em interdependent values}. That is, each agent has her own private signal, and the valuation of each agent is a known function of all n private signals. This captures settings such as valuations for artwork, oil drilling rights, broadcast rights, and many more. 


%In the interdependent value setting, all previous work has assumed a so-called {\sl single-crossing condition}. Single-crossing means that the impact of agent i's private signal, si, on her own valuation is greater than the impact of si on the valuation of any other agent. It is known that without the single-crossing condition an efficient outcome cannot be obtained. We study welfare maximization for interdependent valuations through the lens of approximation. 


%\citep{eden2018interdependent}
%We show that, in general, without the single-crossing condition, one cannot hope to approximate the optimal social welfare any better than the approximation given by assigning the item to a random bidder. 
%Consequently, we introduce a relaxed version of single-crossing, {\sl c-single-crossing}, parameterized by c≥1, which means that the impact of si on the valuation of agent i is at least 1/c times the impact of si on the valuation of any other agent (c=1 is single-crossing). Using this parameterized notion, we obtain a host of positive results. 


% The first step of our procedure maps to the setting of 
% https://www.sas.upenn.edu/~apostlew/paper/pdf/auctionrevised.pdf
%and more generally
% TE2015 



\subsection{Summary and Outline}

%TODO: garrett says more clear explanation. make explicit pointer to background section, and make paragraph of sentence.
%The core of the problem which emerges upon using the machinery of prediction markets in the decision setting is that

The rest of the chapter is structured as follows.
We first introduce a formal model and notation, and conditions under which a simple and practical mechanism built around a second price auction is efficient.
We then present a general direct mechanism for a broader class of information structures, show a condition on valuation profiles under which it is truthful and efficient as well as explore it's limits. 
We then consider a practical variations of the procedure, which remove the need for the mechanism to have detailed of valuations, and consider necessary conditions for their efficiency in equilibrium.  

\section{Model}

As in the previous chapters, there is a single subject that seeks advice from $n$ experts on what action to take from some finite set of alternatives $\cA$. Let $c_i \in\cA \times n $ be the action that was given as advice by expert $i$. Let $a\in\cA$ the action that the subject actually takes.  

Each expert receives a single signal $s_i\in S_i$ which is known only to expert $i$.
Denote a signal profile as $\vec{s}=(s_1,s_2,\ldots,s_n)$.
Let $\vec{s}_{-i}$ denote all signals but $s_i$, and let $(s_i',\vec{s}_{-i})$ denote the profile $\vec{s}$ where $s_i$ has been replaced with $s'_i$.
Similarly, let  $(s_i,\vec{s}_{-i}')$ denote the profile $\vec{s}$ where $s_i$ is fixed and $\vec{s}_{-i}$ been replaced with $\vec{s}_{-i}'$.


Each possible signal profile $\vec{s}$ corresponds to an underlying state of the world; this includes inherent physical properties of both the subject and the actions available to them, as well as the subject's probability for choosing a given action response to different advice by different experts.

There is a joint probability distribution over signals, actions conditional on advice from experts, and rewards. This distribution is common knowledge among the experts. All expectations are with respect to this distribution. 

%NOTE: we could make the C space be the dot product of the Experts and Actions (so that notation like c_i and c_j makes more sense, since the effect of the advice is always potentially contingent on who gives it. 


Since the subject can be influenced differently by different experts who provide the same advice (that is experts can differ in how persuasive they are).
The reward $r$ that the subject receives depends on their chosen action $a$ and the underlying state of the world as determined by the signal profile $\vec{s}$. Conditional on the chosen action $a$ it does not depend on the advice it received, that is the advice can only affect the reward by altering the choice of action.
Note, it is not be the case that the optional advice is always the optimal action to be taken (for a given the signal vector). For example, an expert who knows the subject would find the optimal action unpersuasive may opt to advice for a more persuasive second best action that has a high chance of being actually taken.
Each agent has a reduced form  value function  $v_i: \times_i \ S_i \rightarrow \mathbb{R}_{\geq 0}$, which maps every signal profile of the $n$ agents to the  linear share $\alpha$ of the expected reward $r$ given the bundle of rights is assigned to expert $i$ and they provide optimal advice for the signal profile. 

%TODO how to distingish the value when the expert gives the optimal advice conditional on some other vector of signals (say the reported one, or just observing their element).

$$
V_i(s) = \alpha \expec \left[r \mid \vec{s} \right]
$$


%The valuation functions are monotone non-decreasing in every component.%\footnote{Both of these assumptions are without loss of generality; the space of potential signals can alwyas be , while the ordering of the signals is arbitrary, so they can always be sorted into monotone non-decreasing.}
%TODO monotonicity etc requires an ordering on the S_i, maybe give type...


Each expert reports a signal $b_i\in S_i$, and the vector of reported signals is denoted $\vec{b}=( b_1, b_2, \ldots, b_n)$.
Without loss of generality, assume $S_i=\{0,1,\ldots,q_i\}$.



Mechanisms are pairs $(x,p)$, where $x=(x_1,x_2,\ldots,x_n)$ is a set of allocation functions and $p=(p_1,p_2,\ldots,p_n)$ is a set of payment functions.
The allocation functions $x_i:\times_j \ S_j\rightarrow [0,1]$ map a bid profile $\vec{b}$ to the probability that expert $i$ gets allocated. They hence satisfy $\sum_i x_i(\vec{b}) \leq 1$ for all possible $\vec{b}$.
The payment rules $p_i: \times _j S_j \rightarrow \mathbb R$ map the reported signals $\vec{b}$ to the expected payment from bidder $i$.

Experts are risk neutral, so their expected utility is quasilinear, given in the reduced form by $x_i (\vec{b}) \cdot v_i(\vec{s}) - p_i(\vec{b})$ where $\vec{s}$ is the true signal profile of the experts.

%TODO \expec \left[r \mid v_i(s_i) \right]



%As in \citep{mclean2004informational} an experts  information may be of two qualitatively different kinds: information about the objective characteristics of the subject and the effects of their alternative decisions and how generally persuasive they are to find the recommended action.
%Idiosyncratic information about the agent himself: their ability to aggregate the signals efficiently, and how persuasive the would result as the advisors who selected $c$ in the second stage of the game. 
%The former is of interest to other agents—and consequently is the cause of the interdependence of agents’ values—while the latter is irrelevant to other agents in calculating their values.


Advice is termed optimal for the expert if the expected value of the reward conditional on the chosen expert $i$, their advice  $c_i$  and the true signal vector $\vec{s}$, is maximized. An advice mechanism is efficient if it leads to optimal advice over all potential experts given any potential the true signals $\vec{s}$.

Formally, an advice mechanism is efficient when $c^*_i$ is such that for any other expert $j$ and any advice $c^{'}_j$,

$$
\expec \left[r \mid \vec{s}, c^*_i \right]
\geq
\expec \left[r \mid \vec{s}, c^{'}_j \right]
$$


 
\section{Advice Auctions with Privately Informed Experts}

When experts signal is sufficiently rich, in the sense that it is fully informative of both what their optimal advice is wand what reward to expect from it, a  strong notion of truth telling, dominant strategy, is possible for an advice auction. In the dominant strategy it is in every agent $i$'s best interest to report her true signal $b_i = s_i$ for any possible vector of reports of other agents. 
A second price sealed bid auction for the right to provide the advice and receive a linear share $\alpha$ of the reward, results in a dominant equilibrium where the advice provided is efficient. 

\begin{mech}\label{mech:share2ndprice}[Second Price Auction for Advice and Reward Share (SPAAR)]

   Each expert places a bid for the value of receiving the rights bundle.
   The first part of the mechanism gives the rights bundle to the expert $i^*=\argmax_j \{\vec b\} $ with the highest  bid (a randomly picked one of them, if there are several).
   This expert $i^*$ then provides their advice $c$. 

   That is, the allocation rule is:

   $$x_i(\vec{b}) = \begin{cases} 1 & \text{if } i = \argmax_j \vec{b}\}  \\ 0 & \text{otherwise.} \end{cases}$$
      %TODO did i do this right?

The experts that were not selected receive no payment, while the selected expert $i^*$ receives her share $\alpha$ of the reward $r$ minus the value of the second highest bid.
More formally, given $\vec{b}_{-i^*}$ (the bids for all agents except $i^*$), the payment rule is:

%$p_i= \alpha r - v_i(\vec{s}_{-i},b_i^*)$
   $$p_i(\vec b)= \begin{cases}  \alpha r - \max \vec{b}_{-i}  & \text{if }b_i > \max \vec{b}_{-i} \\ 0 & \text{otherwise.} \end{cases}$$
      %TODO did i do this right?

\end{mech}

A valuation profiles satisfies a private value condition when each experts $i$ signal $s_{i}$ contains all the information in the $\vec{s}$ for them to give their optimal advice (that is maximize $r$ subject to them being the expert providing advice), and know what reward to expect from doing so. Thus there is no further information that is relevant to their choice of action or their expected reward if they are given the rights bundle in the signals received by the other experts  $\vec{s}_{-i}$. Formally,

\begin{defn}[Private Value Condition]\label{PrivateCondition}

For a fixed expert $i$ with signal $s_i$ and their optimal advice $c^*_i$, and for any set of other experts  signals $\vec{s}_{-i}'$:

$$ \expec \left[r \mid \vec{s}, c^*_i \right]
=
\expec \left[r \mid (s_i,\vec{s}_{-i}'), c^*_i \right]
$$

This condition can be re-stated in terms of the reduced form valuation profiles as:

A valuation profile $v_i(\vec{s})$ is said to satisfy the private value condition if for every expert $i$, for any fixed $s_i$, and for any other expert's  signals $\vec{s}_{-i}'$
	
	$$ v_i(\vec{s}) = v_i((s_i, \vec{s}_{-i}')) $$
\end{defn}

This makes the valuation of the rights bundle match the private values condition of \citet{vickrey1961}. The condition is very strong, but the exact match it provides to the natural private value goods setting makes an instructive starting point.
A more formal proof for the existence of an efficient weakly dominant strategy equilibrium for the second price auction wiht private values can be found in \citet{krishna2009auction} Proposition 2.1.

%but arises naturally in the setting where each expert is a specialist in a different treatment. Their examination of a patient combined with their past experience gives them a good estimate of the likely outcome of their specialist treatment, but knowing this does not add to the other experts information. Further since the expert is known to be a specialist in a given treatment they are likely to be unpersuasive as the advisors for other treatments. Thus given what they observe in the examination of the patient (their signal) they know what to expect the outcome to be if they recommend their optimal action, and they expect to be ignored and the default action and outcome of the subject (which is known to all in expectation) to be carried out.  

\begin{prop}\label{prop:2ndefficient}
Given a valuation profile that satisfies the Private Value Condition there is a weakly dominant strategy equilibrium of the Mechanism~\ref{mech:share2ndprice} that results in efficient advice.
\end{prop}



\begin{proof}
The first part of the mechanism is a second price sealed bid auction with private values. 
Given the Private Value Condition (Definition~\ref{PrivateCondition}) if they have the highest bid the payoff to expert $i$ is their expected share of the reward conditioned on their signal $s_i$, which is identical to giving their optimal advice conditioned $\vec{s}$, minus the second highest bid.
If they bid bellow their expected reward it does not change their payoff when they win the auction, but it does reduce the set of states of the world were they are awarded the rights bundle, but in all of them this has expected profit. If they bid above their value it only increases the set of states of the world in which they are awarded the rights bundle but only in those states of the world where it has negative profit. Since there are no further repercusions in the mechanism from their bid, it is dominant strategy to bid their expected value. 

The expert $i$ who places the highest bid and is awarded the rights bundle then maximizes their payoff by selecting their advice $c^*_i$ which maximizes the reward given $\vec{s}$, and by (Definition~\ref{PrivateCondition}) they have all the information in their signal $s_i$ to do so. 
If there was an expert $j$ who could provide advice $c^{'}_j$ that obtained a higher reward they would have had a higher valuation $v_j$ and won the second price auction in the first part of the mechanism. Thus, for any $c^{'}_j$. :

$$
\expec \left[r \mid \vec{s}, c^*_i \right]
\geq
\expec \left[r \mid \vec{s}, c^{'}_j \right]
$$


\end{proof}
% (since ).

The condition can be substantially weakened, since the value of the rights bundle only needs to be private (in the sense that other experts signals are not further informative) for the highest valuation expert $i*$. This is a much more natural condition than it being private value to all experts. It emerges naturally when the evidence base the experts have access to is common, so the signals are only encoding internal knowledge of the experts. The most knowledgeable expert might thus be sufficiently informed that even when observing the reasoning of the others, it would not change their diagnosis or estimate of the right course of action. 


\begin{defn}[Sufficiently Informed Best Expert Condition]\label{SufficientBestCondition}

There is a highest value expert $i$ with signal $s_{i}$, and their optimal advice $c^*_i$, such that for any set of other experts  signals $\vec{s}_{-i}'$ and for any other experts $j \neq i$ advice $c^*_j$:

$$ 
\expec \left[r \mid \vec{s}, c^*_i \right]
=
\expec \left[r \mid ( s_{i^*},\vec{s}_{-i}'), c^*_i \right]
>
\expec \left[r \mid ( s_{i^*},\vec{s}_{-i}'), c^*_j \right]
$$


In terms of the reduced form valuation profiles:

A valuation profile $v(\vec{s})$ is said to satisfy the sufficiently informed best expert condition if  there is a  highest valued expert $i$, such that for all $j$: $$v_i(\vec{s}) > v_j(\vec{s})$$,
and for the fixed $s_i$, and for any other expert's  signals $\vec{s}_{-i}'$
	
	$$ v_i(\vec{s}) = v_i((s_i, \vec{s}_{-i}')) $$
\end{defn}

Note that this is equivalent to the  Private Value Condition for the most valuable expert, and imposes no structure on the valuation profiles of other experts other that they be lower than the highest expert. 

\begin{thm}\label{thm:2ndefficient}
Given a valuation profile that satisfies the Sufficiently Informed Best Expert Condition there is an ex-post efficient Nash Equilibrium of the Mechanism~\ref{mech:share2ndprice} that results in efficient advice.
\end{thm}

\begin{proof}
For the highest valuation bidder nothing has changed relative to the private values setting so their dominant strategy in the auction and their actions once awarded the rights bundle remain the same as in the private value condition. That is their bid is their value $b_i = v_i(s_i)$.

For any other agent $j$ there are a multiplicity of equilibrium strategies that result in the same efficient allocation of the rights bundle in  (but different payments to the highest value expert): any bid $b_j < b_i$ is an ex-post Nash Equilibrium. They are never assigned the rights bundle and always receive payoff 0. If they bid at or higher than $b_i$ they might be assigned the bundle and since by  Definition~\ref{SufficientBestCondition} their value is lower than this, would obtain a negative payoff. 
Thus in equilibrium their bid is bellow $b_i$ and they are never awarded the good. For concreteness consider consider the equilibrium of the auction in which $b_j = v_j(s_j)$.

As in the private value case, the expert $i$ who places the highest bid and is awarded the rights bundle then maximizes their payoff by selecting their advice $c^*_i$ which maximizes the reward given $\vec{s}$, and by (Definition~\ref{SufficientBestCondition}) they have all the information in their signal $s_i$ to do so. 
If there was an expert $j$ who could provide advice $c^{'}_j$ that obtained a higher reward they would have had a higher valuation $v_j$ and by definition won the second price auction in the first part of the mechanism. Thus, for any $c^{'}_j$:

$$
\expec \left[r \mid \vec{s}, c^*_i \right]
>
\expec \left[r \mid \vec{s}, c^{'}_j \right]
$$

\end{proof}



%TODO find a place for this:
%%The opposite situation to the private values setting is where the value  of the reward to the experts is common to all of them. This arises naturally when the subject are oblivious to the advice, that is they will not follow any advice, and thus thus the chosen action $c_i$ or what expert is providing the advice does not matter. In this case the expected reward of all experts is the same. Notice that in this setting there is no question of efficiency, as which experts provides the advice or which advice they provide is irrelevant for the reward. 


\section{Interdependent Valuations And A Direct Reward Sharing Mechanism}

The previous mechanism can be generalized beyond private values or sufficiently informed best expert conditions by replacing the second price sealed bid with a  generalized VCG mechanism \citep{maskin1992auctions} for the initial stage that assigns the rights bundle to the expert, and then allowing the expert to observe the reported bid vector before selection the action $a$. This mechanism is \emph{direct} in the standard sense that agents report their signals. Note these mechanisms are no longer an auction, and while conceptually simple, the mechanism makes direct use of the value function of all agents.

The core of the mechanism is simple. Since there is knowledge by the mechanism over the value function for a given vector of signals, it can use the reported signals to select the highest value expert. The net payment to that expert is then just her share of the reward minus her value at the lowest signal she could have misreported and still obtained the allocation give the other reports. More formally:

\begin{mech}\label{mech:shareVCG}[Direct Reward Share VCG (DRSVCG)]
   The first part of the mechanism gives the rights bundle to the expert $i^*=argmax_j \{v_j(\vec b)\} $ with the highest valuation under the reported signals (a randomly picked one of them, if there are several).

It lets the expert $i^*$ observe  $\vec{b}$ and then select $c^*$.
The subject then observes $c^*$ and $\vec{b}$, takes their action $a$ and receives reward $r$, which the mechanism observes. 


   That is, the allocation rule is
%$$x(\vec{s}) = i \quad \quad\quad \text{when} \quad\quad\quad x_j(\vec{s}) = \begin{cases} 1 & \text{if } j=i \\ 0 & \text{otherwise.} \end{cases}$$
   $$x_i(\vec{b}) = \begin{cases} 1 & \text{if } i = argmax_j \{v_j(\vec b)\}  \\ 0 & \text{otherwise.} \end{cases}$$
      %TODO did i do this right?


The experts that were not selected receive no payment, while the selected expert $i^*$ receives her share $\alpha$ of the reward $r$ minus her valuation of the lowest bid $b_{i^*}^*$ (the critical signal) that would have still resulted in expert $i^*$ being selected.

More formally, given $\vec{b}_{-i}$ (the bids for all agents except $i$), the {\sl critical signal} for $i$ is
$$b_i^* = \min\{b\in S_i\mid x_i(b,\vec b_{-i})=1\}$$
if this minimum exists (otherwise there is no critical signal for $i$).
The payment rule then is

%$p_i= \alpha r - v_i(\vec{s}_{-i},b_i^*)$
   $$p_i(\vec b)= \begin{cases}  \alpha r - v_i(b_i^*,\vec{b}_{-i})& \text{if }i = i^* \\ 0 & \text{otherwise.} \end{cases}$$
      %TODO did i do this right?

\end{mech}

An allocation function $x_i$ is called {\sl deterministic} if $x_i(\vec{b})\in \{0,1\}$ for all $i$ and all $\vec{b}$.
The generalized direct VCG mechanism is deterministic and prior-free. It is not however detail free, in the sense that it requires the mechanism to have access to the valuation function of all experts. 

In general, one cannot hope for truth-telling to be a dominant strategy for the experts. One expert's misreport can cause other experts to also misreport to compensate. Thus the strongest incentive-compatibility (IC) notion that we can hope for in the general setting is is ex-post Nash Equilibrium. That is, it is in every agent $i$'s best interest to report her true signal $b_i = s_i$ given that all other agents reported their true signals profile $b_{-i} = \vec{s}_{-i}$. Fix a signal profile $\vec{s} \in \times _{j} \  S_j$. For all $b_i \in S_i$ we have $$
x_i(\vec{s}) \cdot v_i(\vec{s}) - p_i(\vec{s}) \geq
x_i(b_i, \vec{s}_{-i}) \cdot v_i(\vec{s}) - p_i(b_i, \vec{s}_{-i})
$$
%TODO should this equation go, too?

We use the equivalent, ex-post notion of individually rational (IR): 
$$
x_i(\vec{s}) \cdot v_i(\vec{s}) - p_i(\vec{s}) \geq 0
$$

As is standard, a mechanism is truthful when it is both incentive compatible and individually rational.

\begin{lem}\label{lem:efficient_pass}
Given a valuation profile resulting in an efficient truthful equilibrium of the initial generalized VCG part of the mechanism, the resulting advice is optimal.
\end{lem}

\begin{proof}
Since the highest valuation agent is selected by construction in the truthful efficient equilibrium of the generalized direct VCG mechanism, and this agent is selfish and has access to the reports of the signal profile. 
Since by assumption the equilibrium is truthful, $\vec{b} = \vec{s}$, so the expert $i$ when selecting the advice $c^8_i$ can effectively condition on $\vec{s}$ , and so maximize their payoff  by maximizing the reward conditioned on  $\vec{s}$.
\end{proof}


It is worth highlighting that the lemma is assuming the generalized direct VCG initial part of the mechanism results in an efficient equilibrium. While such equilibrium exist when the signal structure is suitable they are not necessarily unique. See \citet{krishna2009auction} Example 8.3 where a multiplicity of inefficient asymmetric equilibrium  can exist as well as an efficient symmetric equilibrium. 


\subsection{Efficiency}

The link to auction theory allows to immediately derive results when the information structure of the experts is more complicated, such that they need to consider the signals (as expressed in their reports) of the other experts to decide upon the optimal advice.
%The appropriate auction theoretic framework for this is that of interdependent valuations.  

%TODO define truthfulness
%\begin{defn}[Monotonicity]
%	An allocation function $x_i$ is said to be \emph{monotone} if for every $\vec{b}_{-i}$, $x_i(b_i,\vec{b}_{-i})$ is monotone non-decreasing in $b_i$.
%\end{defn}

%Truthful mechanisms can be characterized as follows as in \citet{roughgarden2016optimal} Proposition 5.1.

%\begin{prop}\label{prop:char-ic}
%	Monotonicity is a necessary and sufficient condition for allocation functions $x$ to be \emph{implementable}, {\sl i.e.}, there exist payment functions $p$ such that the mechanism $(x,p)$ is truthful.  Moreover, an analogue of Myerson's payment identity holds, so the payment is uniquely determined by the allocation function.%TODO give reference to Myerson.
%\end{prop}

%It follows that constructing a truthful mechanism is equivalent to constructing a monotone allocation function.
%For deterministic truthful mechanisms, the payment identity of \citet{roughgarden2016optimal} implies the following about the cost charged to a chosen expert. % \citep{eden2018interdependent}.

%\begin{prop}\label{prop:deterministic_payment}
%	Let agent $i$ be the allocated winner at report profile $\vec{b}$ in a deterministic truthful mechanism. Then her cost is her value at the critical signal.
%\end{prop}

%TODO reference proof

A single-crossing condition captures the idea that bidder $i$'s signal has a greater effect on experts $i$'s value than on any other expert's value. We follow the definition in \citet{eden2018interdependent}:

For $s_i = 1, \ldots, k_i$, define $$\frac{\partial v_j(s_i, \vec{s}_{-i})}{\partial s_i} = v_j(s_i, \vec{s}_{-i}) - v_j(s_i - 1, \vec{s}_{-i})$$
%TODO what is k_i?

%We note that there also exists a weaker notion of single-crossing that only requires the inequality to hold at $\vec{s}$ for the bidder $i$ with the highest value, where $i \in \argmax_{k} v_k(\vec{s})$.
\begin{defn}[Single-Crossing]
	A valuation profile is said to satisfy the single-crossing condition if for every expert $i$, for any set of other expert's  signals $\vec{s}_{-i}$, and for every expert $j$, $$\frac{\partial v_i(s_i, \vec{s}_{-i})}{\partial s_i} \geq \frac{\partial v_j(s_i, \vec{s}_{-i})}{\partial s_i}.$$
\end{defn}


\begin{lem}\label{lem:singlecross}
	There is a truthful and efficient ex-post Nash equilibrium of the generalized VCG part of the DRSVCG mechanism when valuation profiles satisfy the single-crossing property.
\end{lem}


A complete proof for this can be found for example in Proposition 10.1 of \citet{krishna2009auction}.

%TODO you define single-crossing to be a property of valuation functions, not signals... no of valuation profiles

%\begin{proof}
%Allocating to the bidder with the highest value is a monotone allocation rule, and therefore, according to Proposition~\ref{prop:char-ic}, it is implementable. The cost for the rights bundle of the chosen expert is then just their value at their critical signal, which is the corresponding payment.
%\end{proof}

Further, one cannot do better than this, since monotonicity of the allocation rule is necessary for an efficient and truthful mechanism. Hence, without single-crossing, it is impossible to have a truthful direct mechanism in general.
The single crossing condition is needed so that that the ex post values of different bidders have the same order as their signals.

\begin{thm}
    When valuation profiles satisfy the single crossing property the DRSVCG there is an efficient ex post equilibrium that results in efficient advice. 
\end{thm}

\begin{proof}
    From Lemma~\ref{lem:singlecross} we have that there is a truthful efficient ex post equilibrium. Given Lemma~\ref{lem:efficient_pass} this implies the advice in this equilibrium is efficient.
\end{proof}

This procedure for a direct advice elicitation mechanism based on the advice auction procedure was here instantiated using the generalized VCG of \citet{maskin1992auctions}  as the first stage auction, but the procedure is generic. It could be, for example, instantiated instead with the randomized mechanism of \citet{eden2018interdependent}, and would obtain the approximation properties that algorithm provides in auctions in our advice setting.


\subsection{Practical Mechanisms} 

The mechanism having access to the value functions is highly impractical in general. Further, direct mechanisms  in the context of motivating experts to provide decision making advice are also highly impractical as experts signals might not be practical to report. It is easy to consider situations in which experts having difficulty understanding each others information, much less having access to a well defined function that is able to incorporate all of it and map it to both optimal advice and a valuation conditional on that advice.
It is much more practical for experts experts to place bids on their value of the rights bundle, rather than directly reporting their signals. This section considers practical advice auctions beyond private values. 

Access to the value function and the reported signals allows the direct VCG mechanism to select the expert $i$ with the highest valuation. Conditions on the valuation profiles for efficiency of the VCG mechanism are  sufficient for an efficient advice mechanism.
This is not the case once the mechanism does not have access the reports and valuation function. That is, even under conditions on the valuation profile such that the initial part of the mechanism would result in an efficient ex post equilibrium, the resulting advice auction need not be efficient. The reason for this is efficient advice requires allocation to the expert with the highest valuation $v_i$, but this value depends on the expert being able to provide optimal advice given the signal $\vec{s}$. Once the signals are not reported to the mechanism, the mechanism cannot reveal them to the expert before it chooses the advice. This suggests a further condition on the equilibrium bids reported that is necessary for an efficient advice mechanism: that there is enough information in the bids reported to the mechanism that when the highest valuation agent is allocated the rights bundle is presented with the bids before providing the advice he is still able to provide the optimal advice.

\begin{defn}[Informative-Bids]
	A bid vector $b$ is said to satisfy the informative-bids condition if for the highest valuation agent $i$  observing the bid vector $b$ from the initial auction stage and their signal $s_i$ the advice $c^*_i$ that maximizes expected reward is equal to when they observe the full signal vector $\vec{s}$. More formally, for any fixed signal profile $\vec{s} $  
	
$$
	 \argmax_{c^{*}_{i}}  \expec \left[r \mid \vec{s}, c^{*}_{i} \right] = \argmax_{c^{*}_{i}}  \expec \left[r \mid s_i, \vec{b}, c^{*}_{i} \right]
$$	
\end{defn}

%While this is necessary for efficiency of the advice auction, it is not clear if it is sufficient.

This condition is unsatisfactory, in that bids result from a specific equilibrium of a valuation profile and a mechanism, and thus are not primitive. What condition on the valuation profiles and mechanism is needed to satisfy this condition is an open problem. 

%\subsection{English Advice Auctions}

Note that mechanism based around sealed bids are impractical to express interdependence of valuations, since to do so the experts would have to place bids contingent on the submitted bids of the other experts. A practical structure that has been substantially studied is using an open ascending auction (also known as an english auction). The price of the rights bundle would rise and experts can irrevocably drop out until there is only one left, who is then awarded the rights bundle, and pays the price at which the last expert dropped out. In the private values case this is equivalent to a second price auction. Efficiency for such auctions requires valuation profiles obey an average crossing condition, and the advice auction around them would also require that the resulting bids  satisfy the informative bids condition.
There are naturally occurring situations where english auctions result in inefficient outcomes,   \citet{hernando2018inefficient} propose a two step mechanism that is still very practical to address such situations, and could be used as the underlying auction for the bundle of advice and reward shares. 

\section{Conclusion}

This chapter presents a bundle of rights approach on  incentives for decision elicitation from multiple experts, which naturally suits an auction approach.
It explores two information structures, one equivalent to private values and a relaxation where only the highest value expert needs to have private values, under which a simple and practical auction based mechanism is efficient. 
The generalized VCG mechanism of \citep{maskin1992auctions} to study the theoretical possibilities in a direct incentive compatible mechanism, resulting that efficiency requires experts values for the rights bundle have a single crossing property.
It then explores  the limits in trying to make efficient practical advice auctions beyond the private values and it's best expert relaxation, in that they require further conditions than those necessary for the underlying initial auction to be efficient.
In this sense advice auctions can be seen as an effective way to elicit non-binding advice in situation where the main concern is identifying the single expert whose signal is most informative.  They appear much more limited as ways of aggregating information dispersed among experts. 

%We then explore two practical variations of the VCG mechanism that relax the assumption that the mechanism cannot access the value functions of experts and that signals cannot be transmitted between experts. We give sufficient conditions on the structure of the signals so that the variations of the mechanisms preserve truthfulness and efficiency.
%TODO should this not be the introduction rather than the conclusion of this section?

