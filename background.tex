\chapter{Background and Related Work}
\label{cha:background}

\epigraph{Desvar\'io laborioso y empobrecedor el de componer vastos libros; el de explayar en quinientas p\'aginas una idea cuya perfecta exposici\'on oral cabe en pocos minutos.}{Jorge Luis Borges, Pr\'ologo de Ficciones.}
%TODO what?

% % from PLnG PredBook: Perdition, burning, and flames (a hell of a book)	
% ...beware of mathematicians, and all those who make empty prophecies. The danger already exists that the mathematicians have made a covenant with the devil to darken the spirit and to confine man in the bonds of Hell.    St. Augustine, De Genesi ad Litteram libri duodecim. Liber Secundus, 17, 37.



Learning what action to take to maximize a reward is a fundamental problem in decision theory.
We first provide an overview of the game theoretical background that underpins all aspects of this work.
% The literatures on bandits and decision markets have remained largely separate, 

We then present the building blocks from the two branches, bounded regret bandit algorithms and mechanism design, that are used in our contributions.
We finalize the chapter by giving an overview of related work that we do not build upon, but which nonetheless informs or motivates our analysis, most notably the analysis of results from randomized controlled trials in the medical literature. 

\section{Notation and Conventions}


The notation used for this work is a compromise to accommodate to as great an extent as possible the conventions of both the bandit algorithms and mechanism design literature, while adding the distinction between the action the algorithm or mechanism chooses and the one that the subject actually carries out in the world. We refer to the former throughout as the \emph{chosen} action and notate it as $c$, and we refer to the action that the subject takes in the world as the \emph{actual} action, which we notate as $a$.

We follow the bandit literature and refer to rewards (notated as $r$ throughout) as directly observable after the actual action is taken. This is skipping the mapping of actions to outcomes, and the utility functions which map those outcomes to rewards, which is standard in the mechanism design literature. The reader wishing to move the analysis more explicitly towards the mechanism design tradition can replace the observed rewards with the von Neumann Morgensten utility function of the agent over the realized outcome.
%TODO: spell it out in greek
We treat experts' signals in the manner of features in a contextual bandit problem. To translate this to the formalism of the mechanism design literature, we can consider the cross-product of the agents' signals as defining a set of partitions, with one partition for each value.
%TODO a lot of things i've never heard of in this sentence...
When we speak of two agents' signals being identical, 
%TODO finish this sentence.

%The prior
%When taking expectations when it is not relevant we omit noting the prior $P$ explcitly. 

All models in the thesis use finite action spaces. 
In the strategic setting this guarantees the existence of a Nash Equilibrium. 
Since actions involve reports of signals $s$, this constrains all signals to also be discrete; note that this does not constrain the underlying latent state of the world $u$ to be discrete.


\section{Game Theory}

Statistical learning algorithms can be understood game theoretically as a game between a forecaster and nature. This is particularly natural in the sequential (online) setting, and a framework termed Learning with Expert Advice and (\cite{cesa2006prediction}) provides a unified treatment from a worst case game theoretic perspective of many such learning settings and algorithms.
This literature largely considers the underlying structures to be zero-sum and thus uses a adversarial model of nature to construct strategies that have good worst case properties. In other words, this is game theory in the style of Von Neumann and Morgensten 1948.
%TODO cite properly?
The framework was applied to the setting of sequential experiments initially by Wald, and the bandit formalism was introduced by (\cite{robbins1952some}).

When there are multiple agents beyond nature interacting, as in the case of elicitation from multiple experts for decision making, there are severe limits to what worst case analysis alone can yield. 
In particular, the notions of the Nash Equilibrium( \cite{nash1950equilibrium}) and common knowledge (\cite{aumann1976agreeing} )provide a useful starting point to thinking about such settings, though they leave us with an embarrassment of riches in terms of the potential equilibrium set.

\section{Online Learning}


The central object of analysis in the online learning framework, also known as the learning with expert advice framework, is the \emph{regret} of an algorithm (the forecaster). This is defined as the difference in the cumulative reward between the reward the algorithm gets and the reward that would have been obtained by some benchmark. 
The most common benchmark is that of best fixed action in hindsight, and this is termed \emph{static regret}. 
When we use the term regret without further modifiers in this thesis, we are referring to this notion.
%TODO explain best fixed action in hindsight...?
%TODO maybe this paragraph can live below the explanation of the learning game.

Two main settings appear in the literature for the play of the environment that is carried out: in the stochastic setting an adversary picks a distribution over actions at the start of the game from which i.i.d. draws are later made; in the non-stochastic (oblivious) adversary setting they select a specific sequence of play before the game begins. Both choices are made with knowledge of the strategy of the learner, which is thus necessarily randomized in the non-stochastic adversarial case.

The basic structure of an online learning game is as follows.

For each round:
\begin{enumerate}
\item The environment chooses an action without revealing it
\item The algorithm chooses a probability distribution over the set of N actions and draws
\item The algorithm observes the reward which depends on its realized action and the realized action of the environment
\end{enumerate}

A crucial aspect of online learning is the feedback model, that is what about the received reward is revealed to the decision making process, be it an agent or an algorithm. Two fundamental extremes are full feedback and the bandit setting. In the first, after realizing a reward, the reward that would have been obtained for any other choice of action by the algorithm is also revealed. In the second, only the reward of the chosen action is revealed. More generally, prediction with partial monitoring \cite{cesa2006regret} generalizes this as follows:

\begin{enumerate}
\item The environment chooses an action without revealing it
\item The algorithm chooses a probability distribution over the set of N actions and draws
\item The algorithm receives the reward which depends on its realized action and the realized action of the environment
\item The feedback is revealed to the forecaster
\end{enumerate}

The feedback and loss matrices are known. In the full information setting, the feedback exactly pins down the value of the reward of the algorithm for any possible choice of the algorithm's action on that period; in the bandit setting the feedback pins down the reward only for the taken action.
%TODO what are those matrices?
%TODO this repeats what is written in the paragraph above.


\subsection{Bandit Algorithms}
%TODO what is a bandit algorithm?!

A one-armed bandit is a anachronistic term for a slot machine. A motivating example in the literature is that of a gambler faced with a set of slot machines, each of which has a different expected reward, and the gambler wishes to find a good strategy to eventually concentrate their play in that with the highest payout. Replacing such slot machines with potential medical treatments (in our motivating example) or with potential online advertisements on the web (in the motivation of much contemporary machine learning research). Two lines of analysis of the regret of algorithms in this settings depend on if the payout distribution is assume fixed (the stochastic setting) or is allowed to vary arbitrarily (the adversarial setting). Within the adversarial setting,  variation in the rewards that can be instantiated before the game (non-adaptive) or at each step of the game (an adaptive adversary). 

Three main families of algorithms with theoretical guarantees exist in the literature, and are based on Bayesian, upper confidence bounds and exponential weights. We also explore a  heuristic algorithm, epsilon greedy, which has been observed to often have good empirical performance \cite{kuleshov:14}.

Thompson Sampling is a very natural bayesian strategy first proposed in \cite{thompson:33}. It selects each action with probability equal to its posterior probability of being the best action, given the prior and rewards observed up to that point. It has excellent practical performance characteristics \cite{chapelle2011empirical}. 

A second family of algorithms encountered in the literature are based on Upper Confidence Bounds (UCB) and originate in (\cite{lai:85,katehakis1995sequential,agrawal1995sample}). They play the action with the highest upper bound on its expected value. This embodies the principle of optimism in the face of uncertainty. A finite time analysis of the regret was presented in (\cite{auer:02a}).

A third family, based on exponential weights, offers maximally robust guarantees, in the sense that it has close optimal (minimax) performance with regard to arbitrary non-stochastic underlying sequences of rewards. This becomes useful when we wish to create hierarchical bandits when the original sequence may not be independent and identically distributed (IID); the sequence that results from a bandit algorithm's choices will not be so by construction.

 Our conceptual contributions are agnostic about the specific underlying algorithm that is used, as long a regret guarantee against a adaptive adversary holds. A thorough analysis of bandit algorithms with adversarial regret guarantees can be found in( \cite{bubeck:12, banditalgo2016}). The underlying intuition as to why a adaptive adversary is sufficient is that whatever dependencies are introduced into the sequential process by the compliance aware algorithm cannot be worse than those, since otherwise the adaptive adversary could adapt such a strategy.
 


\section{Mechanism Design}

The central question of mechanism design is how to structure a game so as to incentivize self-interested agents to achieve some objective.
The two central objectives are \emph{efficiency} -- that the sum of utilities be as great as possible -- and \emph{revenue optimal} -- that the principal which runs the mechanism receives maximal net payment.
In our setting, we are interested in efficiency, that is, allocating the right choice of action for the agents. The rest of this section and the later section thus focus on mechanisms in relation to that objective.

Each agent's information is characterized by their \emph{signal}, which allows the agent to narrow down which realization of possible states of the world they are in. In the literature this is often also called an agent's \emph{type}, particularly when it describes a private valuation of a good by that agent.
We say a mechanism is \emph{incentive compatible} when agents reach maximal utility if they report their true signals to the mechanism. 
Without any further assumptions (i.e. without a probability distribution over said states of the world), signals are of limited use beyond situations with a dominant strategy equilibrium.
% TODO this needs a lot of refinement.
%TODO definition of signal is not clear

Ideally one would like to search for mechanisms that are \emph{strictly dominant strategy} incentive compatible. %TODO what is this?
For many objectives of interest, such mechanisms do not exist, and optimal decision elicitation will turn out to be one of them.
It is worth noting that a \emph{weak} dominant strategy mechanism for optimal decision elicitation is trivial: if the payment to the experts is 0 for all possible states of the world, then any action is weakly dominant, including truthfulness. 
For this reason in the substantive chapters we will simply use the term \emph{dominant strategy} and focus on strictly dominant mechanisms. 
%TODO i find the definitions unclear.

A canonical problem in mechanism design is one where each agent has a quasi-linear utility function that depends on the chosen social alternative, on their private signal, and on monetary transfers, but not on the information available to other agents.
This is known as \emph{private values}.
A class of mechanisms known as Vickrey-Clarke-Groves (VCG)  (\cite{vickrey1961,clarke1971,groves1973}) guarantee that truthful revelation of private information is the dominant strategy for each agent; that is, the mechanisms are dominant \emph{incentive compatible}, and the \emph{efficient} decision is taken. 
This holds for arbitrary dimensions and distributions of signals.
Under independence of signal draws between agents, (\cite{jehiel2001efficient}) provide an efficient mechanism for the case where the quasi-linear utility function of an agent can depend on all agents' private signals. 
To maximize the future social welfare in a dynamic setting, variations of the efficient (VCG) mechanism exist for relatively general dynamic settings (\cite{bergemann2010dynamic,parkes2003van,athey2007designing}).

We focus on models where expert information is endowed to the agents and has no cost of acquisition.

Many settings of interest, including ours, do not in general have dominant strategy mechanisms. 
The literature in microeconomics has largely dealt with this by using a probability distribution over signals, and then treats the problem as one of \emph{bayesian} mechanism design.
The designer then seeks to optimize the objective in expectation over this distribution. The agents' incentives are relaxed relative to dominant strategies, to ensure that their actions are a best response in expectation to the distribution of actions of other agents.

In a bayesian game there are three stages of knowledge possessed by the agents:
\begin{itemize}
   \item \emph{Ex ante}: before values are drawn from the distribution, the agents know this distribution but not their own types (or those of others).
   \item \emph{Interim}: after the agents learn their types, but before playing in the game, the agents know their distribution and know that the other agents' types are drawn from the prior distribution conditioned on their own type.
   \item \emph{Ex post}: the game has been played and the actions of all agents are known.
\end{itemize}

A simple but fundamental result in mechanism design is the \emph{Revelation Principle}.
For any mechanism and equilibrium of the mechanism, there exists an incentive compatible mechanism with the same equilibrium.
The reason is that one can wrap the original non-incentive compatible mechanism with a mechanism that takes a report, assumed truthful, and simulates its optimal play in the original mechanism to pick its payments and allocations, thus achieving the same equilibrium but from the truthful reports. 
This holds in a vast range of situations in both the Bayes-Nash and the Dominant Strategy sense of equilibrium. %TODO vast range, or any mechanism??
It however fails to hold in natural settings of optimal decision elicitation, when agents only learn their types over time or when the mechanism designer does not know the prior (and thus can't simulate).
The learning of types over time is inevitable in the learning setting where there is a sequence of subjects, while in one-off markets a common prior over the signal distribution seems almost impossible.




\subsection{Bandit Algorithms as Mechanisms}


A recent and notable exception to assuming that the algorithm in a bandit setting is able to implement any choice it desires is in the mechanism design literature around bandits (\cite{kremer2014implementing,mansour2015bayesian}).
In this setting the principal is a social planner,  considered to be optimizing the welfare across a sequence of agents, and strategically reveals information about past outcomes to incentivize agents to explore.

A closely related literature to the work of this thesis is focused on the incentive properties of bandit algorithms for their subjects.
The study of this problem was initiated in (\cite{kremer2014implementing}).
At each step of the bandit problem, a new agent (subject) must select which arm to pull. 
The  incentive offered by the social planner is the recommended action. The planner does not offer payments for following the recommendation. %how is recommending an action then incentivizing anyone?
This setting is studied in (\cite{mansour2015bayesian}), who provide a generic black box reduction from bandit algorithms with arbitrary context and (extra) feedback to incentive compatible mechanisms.

The setting where payments are offered to the agents at each step is considered in (\cite{frazier2014incentivizing}).
These works assume the central algorithm embodies a benevolent social planner that attempts to maximize social welfare, and focus on the incentives of the subjects.
In contrast to these works, we abstain from subject incentive considerations, and instead focus on how to incentivize those providing the advice of which decision to take.

Another literature that studies bandit problems in a mechanism design framework is called \emph{Strategic bandit models} and focuses on several players facing (identical) copies of the same set of arms. Players can observe not only their own outcome but also that of their neighbors. A good review of this literature, and the broader literature on the interaction between learning and strategic considerations, is in (\cite{horner2016learning}).

\section{Information Aggregation and Incentives}



A literature in economics and particularly mechanism design is centered on when and how information can be aggregated from multiple agents that receive signals about the state of the world, and have various degrees of strategic sophistication in their actions.
A definitive article on the topic with respect to the common prediction market and Arrow Debreu general equilibrium models is (\cite{ostrovsky2012information}), which also provides an excellent overview of the historical literature within economics.
Ostrovsky studies information aggregation in dynamic markets with a finite number of partially informed strategic traders. 
Trading takes place in a bounded time interval and in every equilibrium; as time approaches the end of the interval, the market price of a \emph{separable} security converges in probability to its expected value conditional on the traders' pooled information. If the security is non-separable, then there exists a common prior over the states of the world and an equilibrium such that information does not get aggregated. 


In these models the fact that securities are settled unambiguously implies that the state of the world is eventually observed.
A largely separate literature, motivated by crowdsourcing, considers how to create incentives to elicit information when the underlying state of the world is not observable to the mechanism. 
In the initial mechanism (\cite{prelec2004bayesian,miller2005eliciting}), truth-telling is a strict Bayesian Nash Equilibrium. These mechanisms typically have many other non-truthful equilibria as well, and some of them may pay better than the truth telling equilibrium, motivating agents to coordinate on non-informative equilibria.
A knowledge-free peer prediction mechanism that does not require knowledge of the information structure and can truthfully elicit private information for a set of information structures slightly smaller than the maximal set is proposed in (\cite{zhang2014elicitability}).
(\cite{kong2016framework}) present a framework for information elicitation mechanisms where truth-telling is the highest paid equilibrium, even when the mechanism does not know the common prior.






%Robust incentives A principal needs to make a decision, and contracts with an expert, who can obtain 
%information relevant to the decision by exerting costly effort. \cite{carroll2019robust}



% http://arxiv.org/pdf/1603.07751v1.pdf
% Peer-prediction is a mechanism which elicits privately-held, non-variable information from
% self-interested agents—formally, truth-telling is a strict Bayes Nash equilibrium of the mechanism.
% The original Peer-prediction mechanism suffers from two main limitations: (1) the
% mechanism must know the “common prior” of agents’ signals; (2) additional undesirable and
% non-truthful equilibria exist which often have a greater expected payoff than the truth-telling
% equilibrium. A series of results has successfully weakened the known common prior assumption.
% However, the equilibrium multiplicity issue remains a challenge.
% In this paper, we address the above two problems. In the setting where a common prior
% exists but is not known to the mechanism we show (1) a general negative result applying to a
% large class of mechanisms showing truth-telling can never pay strictly more in expectation than
% a particular set of equilibria where agents collude to “relabel” the signals and tell the truth
% after relabeling signals; (2) provide a mechanism that has no information about the common
% prior but where truth-telling pays as much in expectation as any relabeling equilibrium and
% pays strictly more than any other symmetric equilibrium; (3) moreover in our mechanism, if
% the number of agents is sufficiently large, truth-telling pays similarly to any equilibrium close
% to a “relabeling” equilibrium and pays strictly more than any equilibrium that is not close to a
% relabeling equilibrium.




% complexity theory of algorithmic persuasion
%http://arxiv.org/pdf/1503.05988.pdf



\section{Prediction Markets} 

The closest contact point between the online learning and information elicitation literature is in the fully supervised case.
That is, the information the market is attempting to aggregate is a forecast of the future state of the world that is not contingent on the actions that the market can influence. Thus, at the time of the realization of the event, we can judge not only the forecast obtained but also any other potential forecast that could have been received. 
This contrasts with the bandit setting, where instead of a forecast of the state of the world we seek an action that results in a state of the world that is maximally beneficial.
The equivalence between trading shares and eliciting beliefs from a single agent by the means of scoring rules goes back at least to (\cite{savage1971elicitation}). 
%Hanson (2003), Pennock (2006), and Chen and Pennock (2007) discussed this cor- respondence for the case of MSR. The study of automated market makers goes back to Black (1971a, 1971b), while formal analysis of inventory-based market makers goes back to Amihud and Mendelson (1980).



% %The correspondence between trading shares and eliciting beliefs from a single agent by the means of scoring rules was first noted by Savage (1971), who also provided additional techni- cal details. 
% %

% %Under the basic MSR (introduced by Hanson (2003, 2007), though the idea of repeatedly using a proper scoring rule to help forecasters aggregate infor- mation goes back to McKelvey and Page (1990))
% \cite{mckelvey1990public}
% %Note also that if each player behaves myopically in each period, the prediction that he will make is his posterior belief about the expected value of the security, given his initial information and the history of revisions up to that point, and thus the €œgame€ turns into the communication process of Geanakoplos and Polemarchakis (1982).

Initiating with the equivalence between market scoring rules and regularized follow-the-leader algorithms in (\cite{chen2010new}), a series of follow up works (\cite{abernethy2013efficient, frongillo2012interpreting, hu2014multi, frongillo2015convergence}) map prediction markets to learning algorithms. 
How the wealth (and thus the accuracy of prices) is concentrated between informed trades in a sequence of markets, using a natural if highly specific trader model (Kelly bettors, equivalently log utility maximizers), is studied in (\cite{beygelzimer2012learning}).

The subject's freedom makes no difference in the analysis of the fully supervised setting; since there is no action to take, there is no sense in which a subject may not follow along. To the degree the information these prediction markets surface is not being used by participants in the world in a way that affects it (as the models assume), we can perfectly evaluate how accurate they are regardless of the other agents' reports.
%TODO the last sentence does not parse


\section{Decision Markets}

Can markets be used not just to understand the underlying distribution over future states of the world, but to select which action to take so as to induce the best distribution? %In the vivid image of (\cite{mackenzie2008engine}), a market as \emph{an engine, not a camera}.%TODO one examiner does not like the use of this metaphor. also this is not a full sentence.
The idea of using prediction markets for decision support originates at least as far back as (\cite{berg2003prediction,hanson2002decision}). 
These mechanisms rely on running a prediction market for the outcome variable of interest for each possible action that the decision maker can take, and voiding those markets for the action not taken.

Corporate prediction markets (\cite{ortner1998forecasting,cowgill2015corporate})  are attempts to have corporate insiders trade claims whose payouts is contigent on a event of interest to their company, such the date a product will ship. In some cases the claims are traded via double auction as in stock markets, while in others a automated market maker is used to provide liquidity. 

In (\cite{othman2010decision}), the authors argue that corporate prediction markets do not capture the right problem for their clients. In particular, by focusing on eliciting probabilities about what their effects will be after decisions have been made, they cannot be used to inform those decisions. It then considers the manipulability of a decision market where (in our terminology) the subject seeks to maximize their utility by always selecting the action that the market prices indicate is best, hence following the \emph{max decision rule}. It is then shown that there are no incentive compatible market scoring rules (and thus by equivalence cost function) markets under this decision rule with multiple experts. %TODO incentive compatible market scoring rules markets?
The intuition is elementary: the last expert to trade with the market can force which of the conditional markets will be settled, so they maximize their profits by changing the price that is most incorrect, and lowering the price of all other actions bellow that.
These results are formally generalized in (\cite{chen2014eliciting}), to show that the subject must use a decision rule with full support to create the right incentives in conditional prediction markets that are used for decision support.

The above works all assume that the participants in the markets are only motivated by the payments they receive on the market (as does this thesis). A related line of work in (\cite{boutilier2012eliciting}) considers the case when the expert has an inherent interest in the decision. 

%from chen2014: Other work related to eliciting predictions for decision making has considered external incentives in addition to the market’s intrinsic incentives. Shi et al. [2009] considered a prediction market where experts can affect the future by taking some actions
%and defined principal-aligned scoring rules that incentivized them to only take “helpful” actions. These rules are similar in spirit to the methods we develop in Section 5,
%but in our setting, experts cannot take actions to affect the future except by influencing the decision maker’s action through their predictions or recommendations. More
%recently, Boutilier [2012] discussed decision making with an expert who has its own
%preferences over the decision maker’s actions. In his model, the expert predicts the
%distribution of a discrete random variable and the decision maker selects an optimal
%action based on the expert’s prediction; however, the random variable being predicted
%is independent of the decision maker’s action. Because the expert has preferences over
%the decision maker’s actions, it has incentives to mislead the decision maker to take
%an action that the expert prefers. Boutilier introduced compensation rules that redress
%the expert’s loss of utility for letting its less preferred actions occur to make the expert
%indifferent again. Different from Boutilier [2012], experts in our setting do not have
%preferences over actions, but our model of decision making is more general and allows
%the decision maker’s action to affect the likelihood of outcomes being predicted.
%Some prior work considers settings where experts can incur cost to improve their
%beliefs and studies how to induce an appropriate degree of learning as well as accurate predictions [Osband 1989]. In this article, we do not consider cost of obtaining
%additional information and assume that experts are endowed with their beliefs.


